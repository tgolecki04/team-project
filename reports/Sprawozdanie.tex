\documentclass[12pt,a4paper]{article}
\usepackage[utf8]{inputenc}  
\usepackage[T1]{fontenc}     
\usepackage[polish]{babel}   
\usepackage{amsmath, amssymb}
\usepackage{graphicx}
\usepackage{geometry}
\geometry{margin=2.5cm}

\title{Sprawozdanie z projektu analizy danych}
\author{Jan Kowalski}
\date{\today}

\begin{document}
	
	\maketitle
	
	\section{Wprowadzenie}
	Celem projektu była analiza zbioru danych dotyczącego sprzedaży diamentów. Badaliśmy zależności między różnymi cechami diamentów, takimi jak masa karatowa, kolor i cena.
	
	\section{Opis danych}
	Zbiór danych zawiera następujące zmienne:
	\begin{itemize}
		\item \textbf{carat} -- masa diamentu w karatach,
		\item \textbf{cut} -- jakość szlifu,
		\item \textbf{color} -- kolor diamentu,
		\item \textbf{price} -- cena w dolarach.
	\end{itemize}
	
	\section{Analiza statystyczna}
	Średnia cena diamentów została obliczona według wzoru:
	\begin{equation}
		\bar{X} = \frac{1}{n} \sum_{i=1}^{n} X_i,
	\end{equation}
	gdzie $X_i$ oznacza cenę $i$-tego diamentu, a $n$ liczbę obserwacji.
	
	Korelacja pomiędzy masą a ceną obliczana była przy użyciu współczynnika Pearsona:
	\begin{equation}
		r = \frac{\sum_{i=1}^{n}(X_i - \bar{X})(Y_i - \bar{Y})}{\sqrt{\sum_{i=1}^{n}(X_i - \bar{X})^2 \sum_{i=1}^{n}(Y_i - \bar{Y})^2}}.
	\end{equation}
	
	\section{Wyniki}
	Analiza wykazała silną dodatnią korelację między masą diamentów a ich ceną. Oznacza to, że większe diamenty mają tendencję do wyższej ceny.
	
	Dodatkowo, regresję liniową ceny na masę można przedstawić w postaci:
	\begin{equation}
		\hat{Y} = \beta_0 + \beta_1 X,
	\end{equation}
	gdzie $\beta_0$ to wyraz wolny, a $\beta_1$ współczynnik kierunkowy regresji.
	
	\section{Wnioski}
	Analiza danych pozwoliła stwierdzić, że masa diamentu jest kluczowym czynnikiem wpływającym na jego cenę. W przyszłych badaniach warto uwzględnić także interakcje między jakością szlifu i kolorem.
	
\end{document}
